\documentclass[preprint]{elsarticle}
\usepackage{lineno,hyperref}
\bibliographystyle{elsarticle-num}

\begin{document}
\begin{frontmatter}

\title{An Explainable Machine Learning Model for Parkinson's Disease Detection using LIME on DaTscan Imagery}

\author{Pavan Rajkumar Magesh\qquad Richard Delwin Myloth\qquad Rijo Jackson Tom
\\Dept. of Computer Science and Engineering
\\CMR Institute of Technology}

\begin{abstract}
Abstract is yet to be completed.
\end{abstract}

\begin{keyword}
Machine Learning\sep AI\sep Parkinson's \sep CNN
\end{keyword}

\end{frontmatter}

\section{Introduction}
Parkinson's disease (PD) is a neurodegenerative disorder that affects predominately dopamine-producing (“dopaminergic”) neurons in a specific area of the brain called substantia nigra \cite{cite1}. Dopamine is an organic chemical which functions both as a hormone and a neurotransmitter playing important roles in the brain and body. In Parkinson's disease a patient loses the ability to retain these dopamine-producing neurons which causes a loss of control over any voluntary actions. This disease may lead to motor and non-motor symptoms such as tremors, slowed movement, sleep disorders, posture imbalance, depression etc. \cite{cite2}. There exists a variety of brain scans such as Magnetic Resonance Imaging(MRI), Functional Magnetic Resonance Imaging(fMRI), (Positron Emission Tomography)PET etc, but the Single-photon Emission Computed Tomography(SPECT) functional imaging technique is most widely used in European clinics for early diagnosis of Parkinson's disease \cite{cite3}.

\section*{References}
\bibliography{draft}
\end{document}
